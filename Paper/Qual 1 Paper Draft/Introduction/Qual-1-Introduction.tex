% Options for packages loaded elsewhere
\PassOptionsToPackage{unicode}{hyperref}
\PassOptionsToPackage{hyphens}{url}
%
\documentclass[
  man]{apa7}
\usepackage{amsmath,amssymb}
\usepackage{iftex}
\ifPDFTeX
  \usepackage[T1]{fontenc}
  \usepackage[utf8]{inputenc}
  \usepackage{textcomp} % provide euro and other symbols
\else % if luatex or xetex
  \usepackage{unicode-math} % this also loads fontspec
  \defaultfontfeatures{Scale=MatchLowercase}
  \defaultfontfeatures[\rmfamily]{Ligatures=TeX,Scale=1}
\fi
\usepackage{lmodern}
\ifPDFTeX\else
  % xetex/luatex font selection
\fi
% Use upquote if available, for straight quotes in verbatim environments
\IfFileExists{upquote.sty}{\usepackage{upquote}}{}
\IfFileExists{microtype.sty}{% use microtype if available
  \usepackage[]{microtype}
  \UseMicrotypeSet[protrusion]{basicmath} % disable protrusion for tt fonts
}{}
\makeatletter
\@ifundefined{KOMAClassName}{% if non-KOMA class
  \IfFileExists{parskip.sty}{%
    \usepackage{parskip}
  }{% else
    \setlength{\parindent}{0pt}
    \setlength{\parskip}{6pt plus 2pt minus 1pt}}
}{% if KOMA class
  \KOMAoptions{parskip=half}}
\makeatother
\usepackage{xcolor}
\usepackage{graphicx}
\makeatletter
\def\maxwidth{\ifdim\Gin@nat@width>\linewidth\linewidth\else\Gin@nat@width\fi}
\def\maxheight{\ifdim\Gin@nat@height>\textheight\textheight\else\Gin@nat@height\fi}
\makeatother
% Scale images if necessary, so that they will not overflow the page
% margins by default, and it is still possible to overwrite the defaults
% using explicit options in \includegraphics[width, height, ...]{}
\setkeys{Gin}{width=\maxwidth,height=\maxheight,keepaspectratio}
% Set default figure placement to htbp
\makeatletter
\def\fps@figure{htbp}
\makeatother
\setlength{\emergencystretch}{3em} % prevent overfull lines
\providecommand{\tightlist}{%
  \setlength{\itemsep}{0pt}\setlength{\parskip}{0pt}}
\setcounter{secnumdepth}{-\maxdimen} % remove section numbering
% Make \paragraph and \subparagraph free-standing
\ifx\paragraph\undefined\else
  \let\oldparagraph\paragraph
  \renewcommand{\paragraph}[1]{\oldparagraph{#1}\mbox{}}
\fi
\ifx\subparagraph\undefined\else
  \let\oldsubparagraph\subparagraph
  \renewcommand{\subparagraph}[1]{\oldsubparagraph{#1}\mbox{}}
\fi
\ifLuaTeX
\usepackage[bidi=basic]{babel}
\else
\usepackage[bidi=default]{babel}
\fi
\babelprovide[main,import]{english}
% get rid of language-specific shorthands (see #6817):
\let\LanguageShortHands\languageshorthands
\def\languageshorthands#1{}
% Manuscript styling
\usepackage{upgreek}
\captionsetup{font=singlespacing,justification=justified}

% Table formatting
\usepackage{longtable}
\usepackage{lscape}
% \usepackage[counterclockwise]{rotating}   % Landscape page setup for large tables
\usepackage{multirow}		% Table styling
\usepackage{tabularx}		% Control Column width
\usepackage[flushleft]{threeparttable}	% Allows for three part tables with a specified notes section
\usepackage{threeparttablex}            % Lets threeparttable work with longtable

% Create new environments so endfloat can handle them
% \newenvironment{ltable}
%   {\begin{landscape}\centering\begin{threeparttable}}
%   {\end{threeparttable}\end{landscape}}
\newenvironment{lltable}{\begin{landscape}\centering\begin{ThreePartTable}}{\end{ThreePartTable}\end{landscape}}

% Enables adjusting longtable caption width to table width
% Solution found at http://golatex.de/longtable-mit-caption-so-breit-wie-die-tabelle-t15767.html
\makeatletter
\newcommand\LastLTentrywidth{1em}
\newlength\longtablewidth
\setlength{\longtablewidth}{1in}
\newcommand{\getlongtablewidth}{\begingroup \ifcsname LT@\roman{LT@tables}\endcsname \global\longtablewidth=0pt \renewcommand{\LT@entry}[2]{\global\advance\longtablewidth by ##2\relax\gdef\LastLTentrywidth{##2}}\@nameuse{LT@\roman{LT@tables}} \fi \endgroup}

% \setlength{\parindent}{0.5in}
% \setlength{\parskip}{0pt plus 0pt minus 0pt}

% Overwrite redefinition of paragraph and subparagraph by the default LaTeX template
% See https://github.com/crsh/papaja/issues/292
\makeatletter
\renewcommand{\paragraph}{\@startsection{paragraph}{4}{\parindent}%
  {0\baselineskip \@plus 0.2ex \@minus 0.2ex}%
  {-1em}%
  {\normalfont\normalsize\bfseries\itshape\typesectitle}}

\renewcommand{\subparagraph}[1]{\@startsection{subparagraph}{5}{1em}%
  {0\baselineskip \@plus 0.2ex \@minus 0.2ex}%
  {-\z@\relax}%
  {\normalfont\normalsize\itshape\hspace{\parindent}{#1}\textit{\addperi}}{\relax}}
\makeatother

\makeatletter
\usepackage{etoolbox}
\patchcmd{\maketitle}
  {\section{\normalfont\normalsize\abstractname}}
  {\section*{\normalfont\normalsize\abstractname}}
  {}{\typeout{Failed to patch abstract.}}
\patchcmd{\maketitle}
  {\section{\protect\normalfont{\@title}}}
  {\section*{\protect\normalfont{\@title}}}
  {}{\typeout{Failed to patch title.}}
\makeatother

\usepackage{xpatch}
\makeatletter
\xapptocmd\appendix
  {\xapptocmd\section
    {\addcontentsline{toc}{section}{\appendixname\ifoneappendix\else~\theappendix\fi\\: #1}}
    {}{\InnerPatchFailed}%
  }
{}{\PatchFailed}
\DeclareDelayedFloatFlavor{ThreePartTable}{table}
\DeclareDelayedFloatFlavor{lltable}{table}
\DeclareDelayedFloatFlavor*{longtable}{table}
\makeatletter
\renewcommand{\efloat@iwrite}[1]{\immediate\expandafter\protected@write\csname efloat@post#1\endcsname{}}
\makeatother
\usepackage{lineno}

\linenumbers
\usepackage{csquotes}
\makeatletter
\renewcommand{\paragraph}{\@startsection{paragraph}{4}{\parindent}%
  {0\baselineskip \@plus 0.2ex \@minus 0.2ex}%
  {-1em}%
  {\normalfont\normalsize\bfseries\typesectitle}}

\renewcommand{\subparagraph}[1]{\@startsection{subparagraph}{5}{1em}%
  {0\baselineskip \@plus 0.2ex \@minus 0.2ex}%
  {-\z@\relax}%
  {\normalfont\normalsize\bfseries\itshape\hspace{\parindent}{#1}\textit{\addperi}}{\relax}}
\makeatother

\ifLuaTeX
  \usepackage{selnolig}  % disable illegal ligatures
\fi
\IfFileExists{bookmark.sty}{\usepackage{bookmark}}{\usepackage{hyperref}}
\IfFileExists{xurl.sty}{\usepackage{xurl}}{} % add URL line breaks if available
\urlstyle{same}
\hypersetup{
  pdftitle={Introduction Draft},
  pdfauthor={Jimmy},
  pdflang={en-EN},
  hidelinks,
  pdfcreator={LaTeX via pandoc}}

\title{Introduction Draft}
\author{Jimmy\textsuperscript{}}
\date{}


\shorttitle{SHORT TITLE}

\affiliation{\phantom{0}}

\begin{document}
\maketitle

Social science research increasingly focuses on complex effects (e.g., nonlinear effects, moderation) rather than simple bivariate relationships as the real world is rarely simple and straightforward (Carte \& Russell, 2003; MacKinnon \& Luecken, 2008; Cunningham \& Ahn, 2019). Research demonstrates that exercising may help people lose weight, but people may be further interested in how, when, for whom, and under what conditions that exercising can do for losing weight. Moderation (or interaction) research can answer such questions by investigating how a third variable (or a group of additional variables) modifies relations among variables of interest.

One widespread way to model moderation is through regression model, specifically incorporating an interaction term \(XZ\):
\begin{equation}
Y = b_{0} + b_{1}X + b_{2}Z + b_{3}XZ + \epsilon,
\end{equation}
where \(b_{0}\) is the intercept, \(b_{1}\) and \(b_{2}\) are the regression coefficients for \(X\) and \(Z\), \(b_{3}\) is the coefficient for the interaction term \(XZ\), and \(\epsilon\) is the residual term. To maintain consistency with the naming conventions used by Marsh et al.~(2004), we refer to main effects (i.e., non-interaction effects) as ``first-order effects''. Hence \(X\) and \(Z\) are first-order variables and \(b_{1}\) and \(b_{2}\) are first-order effects in this case. Classical regression assumes variables are measured without error, which may lead to biased parameter estimates (especially for the interaction) when measurement errors are not uncommonly present in empirical research (Bollen, 1989; Cohen et al., 2003; Caroll et al., 2006). To address this problem, researchers use latent variables that are inferred and measured by a set of observed indicators in the structural equation modeling (SEM) framework, which can control and accommodate measurement errors in these observed indicators (Bollen, 2002). For example, depression is widely tested and measured by the Center for Epidemiologic Studies Depression (CES-D) scale consisting of 20 items (Radloff, 1977). Moderation models based on SEM provide reliably true relationships among latent constructs (Mueller, 1997; Steinmetz et al., 2011; Cham et al., 2012; Maslowsky et al., 2015).

The Two-Stage Path Analysis (2S-PA) is a method of modeling latent variables based SEM and it is demonstrate to show the capability of producing parameter estimates with less standard error bias, higher convergence rates, and better handling of Type I error in small samples (Lai \& Hsiao, 2021). Given its good statistical property, we extended the 2S-PA method to incorporate latent interaction estimation in this study, and named it 2S-PA-Int. We reviewed two widely used latent interaction models, Unconstrained Product Indicator (UPI; Marsh et al., 2004) and Reliability-Adjusted Product Indicator (RAPI; Hsiao et al., 2018), and conducted a Monte Carlo simulation study to compare their performance with 2S-PA-Int. To proceed, we first introduced a classical model of latent interaction and then presented UPI, RAPI, and 2S-PA-Int with technical details.

\hypertarget{a-classical-model-of-latent-interaction}{%
\subsection{A Classical Model of Latent Interaction}\label{a-classical-model-of-latent-interaction}}

Kenny and Judd (1984) first proposed a classical structural model that provided a seminal idea of estimating latent interaction effecs, focusing on a basic scenario involving two latent predictors and one latent interaction term:
\begin{equation}
y = \alpha + \gamma_{x}\xi_{x} + \gamma_{m}\xi_{m} + \gamma_{xm}\xi_{x}\xi_{m} + \zeta,
\end{equation}
where \(\alpha\) is the constant intercept, \(\xi_{x}\) and \(\xi_{m}\) denote the first-order latent predictors, and the product \(\xi_{x}\xi_{m}\) constitutes the interaction term. Note that \(\xi_{x}\) and \(\xi_{m}\) are allowed to correlate with each other. As for other parameters, \(\zeta\) is the model's disturbance term assumed to follow a normal distribution \(\zeta \sim N(0, \psi)\) where \(\psi\) is a scalar representing the variance of \(\zeta\) that captures unobserved factors influencing the dependent variable. The coefficients \(\gamma_{x}\) and \(\gamma_{m}\) indicate first-order effects of the latent predictors, whereas \(\gamma_{xm}\) quantifies the latent interaction effect. The dependent variable \(y\) can be either directly observed or a latent construct.

The measurement model for the first-order latent predictors, for instance \(\xi_{x}\), are described by the following confirmatory factor analysis (CFA) framework:
\begin{equation}
\mathbf{x} = \boldsymbol{\tau_{x}} + \boldsymbol{\lambda_{x}}\xi_{x} + \boldsymbol{\delta_{x}},
\end{equation}
wherein, for each indicator \(i = [1, \ 2, \ ..., \ p]\) associated with the latent predictor \(\xi_{x}\), \(\mathbf{x}\) denotes a \(p \times 1\) vector of observed first-order indicators (i.e., indicators of \(\xi_{x}\)); \(\xi_{x}\) is a \(1 \times 1\) scalar representing the latent variable; \(\boldsymbol{{\tau_{x}}}\) is a \(p \times 1\) vector of constant intercepts; \(\boldsymbol{\lambda}_{x}\) is a \(p \times p\) vector of factor loadings, and \(\boldsymbol{\delta_{x}}\) is a \(p \times 1\) vector of indicator-level measurement errors. Each measurement error \(\delta_{x_{i}}\) is normally distributed with a mean of zero and a variance \(\theta_{x_{i}}\). Assuming local independence (i.e., first-order indicators are uncorrelated with each other when indicating the same latent variable), the variance-covariance matrix of all indicators' measurement errors is a diagonal matrix \(\mathbf{\Theta_{\delta_{x}}} = diag(\theta_{x_{1}}, \theta_{x_{2}}, ..., \theta_{x_{p}})\). This measurement model and its associated parameters similarly apply to \(\xi_{m}\).

Kenny and Judd's original formulation of their model omitted the intercept \(\alpha\), a point later corrected by Jöreskog and Yang (1996) who revised the model under a set of assumptions. The revised latent interaction model is grounded in three primary assumptions related to multivariate normal distribution and independence: (1) The measurement errors of first-order indicators, the first-order latent predictors, and the disturbance term in the structural model are multivariate normal, uncorrelated, and independent to each other (i.e., \(Corr[\delta, \xi] = 0\); \(Corr[\zeta, \xi] = 0\); \(Corr[\delta, \zeta] = 0\) where \(Corr\) denotes the correlation index); (2) All measurement errors are mutually independent and uncorrelated to each other (i.e., \(Corr[\delta_{i}, \delta_{i'}] = 0\) for \(i \neq i'\)); (3) The correlation between first-order latent predictors (i.e., \(Corr[\xi_{x}, \xi_{m}]\)) is assumed to be non-zero and freely estimated since \(\xi_{x}\xi_{m}\) may have a non-normal distribution even though \(\xi_{x}\) and \(\xi_{m}\) are normally distributed with means of 0 (Jöreskog \& Yang, 1996).

Algina and Moulder (2001) advanced Jöreskog and Yang's (1996) model by incorporating a mean-centering technique. They used mean-centered first-order indicators (e.g., \(x_{i} - \mu_{x_{i}}\) where \(\mu_{x_{i}}\) is the mean of \(x_{i}\)) to form product indicators (PI) that indicate the latent interaction term, which enhances the modeling approach by improving interpretability of parameter estimates, facilitating model convergence rate, and reducing bias of estimating the interaction effect (Algina \& Moulder, 2001; Moulder \& Algina, 2002; Marsh et al., 2004). Bsides, mean-centering first-order indicators diminishes problem of multicollinearity, clarifying the distinct contributions of first-order latent variable and their interaction (Schoemann \& Jorgensen, 2021).

\hypertarget{unconstrained-product-indicator-upi}{%
\subsection{Unconstrained Product Indicator (UPI)}\label{unconstrained-product-indicator-upi}}

Although Algina and Moulder's modification improved the statistical properties of parameter estimation, their model necessitated implementation of complicated nonlinear constraints in their model. Constraints in tructural equation modeling (SEM) are predefined conditions or restrictions applied to model parameters to ensure model identifiability, theoretical consistency, and interpretability (Kline, 2016). Usual constraints include equality constraints (i.e., equal values constrained on two or more parameters), fixed-value constraints (i.e., specific values constrained on parameters), and nonlinear constraints (i.e., specific relationship between first-order latent variables and their interaction term as constraints). Nonlinear constraints are required in Algina and Moulder's model. Specifically, parameters are constrained that relate PIs to the interaction term (e.g., factor loadings, variances and covariances of error among PIs) and relate PIs to firsr-order indicators (e.g., error covariances between PIs and first-order indicators). Additionally, since the model is based on the assumption that first-order latent predictors have multivariate normal distribution, constraints on variances and covariances between them and their interaction term are required as well.

Marsh et al.~(2004) explored the possibility of removing complex constraints and introduced the pivotal Unconstrained Product Indicator (UPI) method to simplify model specifications and reduce risks of erroneous specification and convergence issue. The structural model of UPI is the same as equation (2) except for omitting the intercept \(\alpha\). To illustrate, consider a measurement model in which the latent variables \(\xi_{x}\) and \(\xi_{m}\) are each associated with three indicators:

\begin{align}
    \begin{bmatrix}
        x_{1} \\
        x_{2} \\ 
        x_{3}
    \end{bmatrix} =
    \begin{bmatrix}
        \tau_{x_{1}} \\
        \tau_{x_{2}} \\ 
        \tau_{x_{3}}
    \end{bmatrix} +
    \begin{bmatrix}
        \lambda_{x_{1}} \\
        \lambda_{x_{2}} \\ 
        \lambda_{x_{3}}
    \end{bmatrix}
    \begin{bmatrix}
        \xi_{x} \\
    \end{bmatrix} +
    \begin{bmatrix}
        \delta_{x_{1}} \\
        \delta_{x_{2}} \\ 
        \delta_{x_{3}}
    \end{bmatrix}, %
    \begin{bmatrix}
        m_{1} \\
        m_{2} \\ 
        m_{3}
    \end{bmatrix} =
    \begin{bmatrix}
        \tau_{m_{1}} \\
        \tau_{m_{2}} \\ 
        \tau_{m_{3}}
    \end{bmatrix} +
    \begin{bmatrix}
        \lambda_{m_{1}} \\
        \lambda_{m_{2}} \\ 
        \lambda_{m_{3}}
    \end{bmatrix}
    \begin{bmatrix}
        \xi_{m} \\
    \end{bmatrix} +
    \begin{bmatrix}
        \delta_{m_{1}} \\
        \delta_{m_{2}} \\ 
        \delta_{m_{3}}
    \end{bmatrix}
\end{align}

Marsh et al.~(2004) proposed two strategies of specifying the UPI model: the all-pair UPI and the matched-pair UPI. In the all-pair UPI, the latent interaction term is indicated by all possible configurations of pairs formed by the first-order indicators of \(\xi_{x}\) and \(\xi_m\):

\begin{align}
    \begin{bmatrix}
        x_{1}m_{1} \\
        x_{1}m_{2} \\
        x_{1}m_{3} \\ 
        x_{2}m_{1} \\
        ... \\
        x_{3}m_{3}
    \end{bmatrix} = 
    \begin{bmatrix}
        \tau_{x_{1}m_{1}} \\
        \tau_{x_{1}m_{2}} \\ 
        \tau_{x_{1}m_{3}} \\ 
        \tau_{x_{2}m_{1}} \\ 
        ...\\
        \tau_{x_{3}m_{3}} 
    \end{bmatrix} +
    \begin{bmatrix}
        \lambda_{x_{1}m_{1}} \\
        \lambda_{x_{1}m_{2}} \\ 
        \lambda_{x_{1}m_{3}} \\ 
        \lambda_{x_{2}m_{1}} \\ 
        ...\\
        \lambda_{x_{3}m_{3}}
    \end{bmatrix}
    \begin{bmatrix}
        \xi_{x}\xi_{m} \\
    \end{bmatrix} +
    \begin{bmatrix}
        \delta_{x_{1}m_{1}} \\
        \delta_{x_{1}m_{2}} \\ 
        \delta_{x_{1}m_{3}} \\
        \delta_{x_{2}m_{1}} \\
        ... \\
        \delta_{x_{3}m_{3}}
    \end{bmatrix},
\end{align}
where each PI is derived from multiplying two corresponding mean-centered first-order indicators, one from \(\xi_{x}\) and the other from \(\xi_{m}\) (e.g., the PI \(x_{1}m_{1}\) is formed by the product of \(x_{1}\) and \(m_{1}\). The coefficients \({\tau_{x_{i}m_{i}}}\), \({\lambda_{x_{i}m_{i}}}\) and \({\delta_{x_{i}m_{i}}}\) are estimated freely as intercepts, factor loadings and measurement errors, respectively. The total number of PIs are the multiplicative product of the number of first-order indicators for each latent predictor. In this case, nine unique configurations are generated (\(3 \times 3 = 9\)).

Regarding the matched-pair UPI, the indicators are matched to create PIs:

\begin{align}
    \begin{bmatrix}
        x_{1}m_{1} \\
        x_{2}m_{2} \\
        x_{3}m_{3}
    \end{bmatrix} =
    \begin{bmatrix}
        \tau_{x_{1}m_{1}} \\
        \tau_{x_{2}m_{2}} \\ 
        \tau_{x_{3}m_{3}}
    \end{bmatrix} + 
    \begin{bmatrix}
        \lambda_{x_{1}m_{1}} \\
        \lambda_{x_{2}m_{2}} \\ 
        \lambda_{x_{3}m_{3}} 
    \end{bmatrix}
    \begin{bmatrix}
        \xi_{x}\xi_{m} \\
    \end{bmatrix} +
    \begin{bmatrix}
        \delta_{x_{1}m_{1}} \\
        \delta_{x_{2}m_{2}} \\ 
        \delta_{x_{3}m_{3}},
    \end{bmatrix}
\end{align}
This alternative formulation of UPI results in a significantly reduced number of PIs due to the straightforward configuration strategy. Marsh et al.~(2004) suggested that the matched-pair UPI is more favorable based on two criteria: (1) It uses all available information by utilizing every first-order indicator; (2) It avoids redundancy by ensuring that no firs-order indicators are used more than once. This method is thus recommended for simplicity and effectiveness. Furthermore, they demonstrated that the matched-pair UPI excels by exhibiting lower bias and increased robustness in estimating the interaction effect, particularly under the violation of normality assumptions.

Since the mean of \(\xi_{x}\xi_{m}\) is not equal to 0 even though \(\xi_{x}\) and \(\xi_{m}\) are assumed to have 0 means with non-zero correlation, Marsh et al.~(2004) included a mean structure in their UPI model: \(\mathbf{\kappa} = (0,\ 0,\ Cov[\xi_{x}, \xi_{m}])^T\), where \(\mathbf{\kappa}\) represents the model's mean structure. In this structure, the means of \(\xi_{x}\) and \(\xi_{m}\) are presumed to be 0, while the mean of the latent interaction term, denoted \(Cov[\xi_{x}, \xi_{m}]\), is constrained as the covariance between \(\xi_{x}\) and \(\xi_{m}\) (see Algina \& Boulder {[}2001{]} for more details). This adjustment ensures that the model accurately reflects the statistical relations between the first-order latent variables and their interaction term.

Lin et al.~(2010) recently proposed a more refined method, the Double Mean Centering (DMC) strategy, and showed its advantages in eliminating the necessity of including a mean structure, simplifying the procedure of model specification, and demonstrating outstanding performance of parameter estimation under the violation of normality assumption. This method begins with mean-centering first-order indicators, and continues to mean-center PIs of the interaction term (e.g., \(x_{i}m_{i} - \mu_{x_{i}m_{i}}\). Therefore we used the UPI method with DMC in this study.

Although UPI has more simplicity and better performance of parameter estimation compared to the classical model, a arbitrariness-complexity dilemma between the all-pair and the matched-pair methods is not well resolved (Foldness \& Hadtvet, 2014). Consider a model with two complex psychological constructs that each may involve over 10 items to achieve sufficient coverage and depth of theory. The all-pair UPI method may potentially lead to a latent interaction term indicated by hundreds of PIs. More items can improve the representation of latent constructs and theoretically increase statistical power for detecting nuanced effects, but also result in a cumbersome model that negatively impacts interpretability, escalates computational demands, and overfits the sample. The matched-pair UPI strategy effectively simplifies model complexity by reducing the number of necessary PIs but introduces a challenge of indicator selection. Researchers may aggregate multiple observed indicators into fewer parcels (Jackman et al., 2011) or prioritize items with higher reliability for PI formation (Wu et al., 2013). However, there is not a consensus on the best strategy to form matched pairs, and the considerable arbitrariness across various alternative approaches introduces uncertainty in selecting the optimal strategy and complicates the decision-making process in model specification. Marsh et al.~(2004) simplifies this process by applying the matched-pair UPI to the model with equal number of first-order indicators, but it is very likely that substantive researchers may need to deal with unbalanced numbers of first-order indicators.

\hypertarget{reliability-adjusted-product-indicator-rapi}{%
\subsection{Reliability Adjusted Product Indicator (RAPI)}\label{reliability-adjusted-product-indicator-rapi}}

The RAPI introduced by Hsiao et al.~(2018) also forms PI, but it uses composite scores (sum or mean scores) of multiple first-order items. Specifically, it combines all first-order indicators into single indicators to indicate first-order latent variables, and forms PIs by multiplying the single indicators to indicate the latent interaction term. Accordingly, the formed PI is a single indicator as well. This method effectively circumvents the issue of arbitrariness in indicator selection while using all information without redundancy. RAPI adjusts for measurement error in composite scores by constraining error variances of single indicators, thus ensuring that parameter estimates are less biased. The model is succinctly represented as follows:
\begin{align}
    \begin{bmatrix}
        x_{comp} \\
        m_{comp} \\
        x_{comp} \cdot m_{comp}
    \end{bmatrix} = 
    \begin{bmatrix}
        \tau_{x_{comp}} \\
        \tau_{m_{comp}} \\ 
        \tau_{x_{comp} \cdot m_{comp}} 
    \end{bmatrix} + 
    \begin{bmatrix}
        1 & 0 & 0 \\
        0 & 1 & 0 \\ 
        0 & 0 & 1 
    \end{bmatrix}
    \begin{bmatrix}
        \xi_{x} \\  
        \xi_{m} \\ 
        \xi_{x}\xi_{m}
    \end{bmatrix} +
    \begin{bmatrix}
        \delta_{x_{comp}} \\
        \delta_{m_{comp}} \\ 
        \delta_{x_{comp} \cdot m_{comp}}
    \end{bmatrix},
\end{align}
where \(x_{comp}\) and \(m_{comp}\) are the composite scores formed by their corresponding first-order indicators, and \(x_{comp} \cdot m_{comp}\) is the formed PI indicating the latent interaction term. These composite scores serve as single indicators for their respective latent variables, with factor loadings uniformly constrained to \(1\). The measurement errors are represented by \(\mathbf{\delta}\).

An important feature of the RAPI method is that it can account for measurement error within first-order indicators by including error-variance constraints computed using composite reliability. Although technically composite reliability estimates as part of error-variance contraints can be obtained by any existing methods, Hsiao et al.~(2021) summarized and compared four normally used estimators for composite reliability: Cronbach's \(\alpha\) (Cronbach, 1951), \(\omega\) (McDonald, 1970; Raykov, 1997), the greatest lower bound reliability (GLB; Berge \& Sočan, 2004), and Coefficient H (Hancock \& Mueller, 2001). Suppose that \(\rho_{xx'}\) denotes the estimated reliability index, the error variance of \(\xi_{x}\) can be shown as a function of the reliability index:
\begin{equation}
\hat{\sigma}^2_{\delta_{x}} = (1 - \rho_{xx'})\hat{\sigma}^2_{{x}},
\end{equation}
where \(\hat{\sigma}^2_{\delta_{x}}\) represents the sample-estimated error variance and \(\hat{\sigma}^2_{{x}}\) represents the sample-estimated variance of the indicator. The formula can be converted by linear transformation to show the relations between variances of the error and the latent predictor in terms of reliability: \(\hat{\sigma}_{\delta_{x}}^2 = [(1 - \rho_{xx'})/{\rho_{xx'}}]\hat{\sigma}^2_{\xi_{x}}\), where \(\hat{\sigma}^2_{\xi_{x}}\) represents the estimated variance of \(\xi_{x}\) and \(\hat{\sigma}^2_{{x}} = {\hat{\sigma}^2_{\xi_{x}} + \hat{\sigma}^2_{\delta_{x}}}\). Hence, under the assumption of independently and identically distributed measurement error, the equation for the error-variance constraint of the interaction term \(\xi_{x}\xi_{m}\) can be derived:
\begin{equation}
\begin{aligned}
\hat{\sigma}^2_{\delta_{xm}} = & \rho_{xx'}\hat{\sigma}^2_{{x}}(1 - \rho_{mm'}\hat{\sigma}^2_{{m}}) + \\&
                        \rho_{mm'}\hat{\sigma}^2_{{m}}(1-\rho_{xx'})\hat{\sigma}^2_{{x}} + \\&
                        (1 - \rho_{xx'})\hat{\sigma}^2_{{x}}(1 - \rho_{mm'})\hat{\sigma}^2_{{m}}. 
\end{aligned}
\end{equation}
More technical details are available in Appendix A of Hsiao et al.~(2018).

The utilization of composite scores as single indicators significantly simplifies model specification, as the total number of PIs directly corresponds to the number of interaction terms. By accounting for measurement error, RAPI is expected to produce less biased estimates of interaction effects and exhibit enhanced statistical power. However, the method's effectiveness is contingent upon accurate estimation of reliability measures since inaccurate reliability estimates which serve as the basis for error constraints can lead to biased results. Despite its acceptable model complexity and approachable implementation, Hsiao et al.~(2021) showed that RAPI may lead to non-positive definite matrices due to negative error variance and inflated interaction effect estimates, under conditions of low reliability (e.g., r = .70) and small sample size (e.g., N = 100). This suggests that RAPI may generate unstable interaction estimates under such conditions.

\hypertarget{two-stage-path-analysis-with-interaction-2s-pa-int}{%
\subsection{Two-stage Path Analysis with Interaction (2S-PA-Int)}\label{two-stage-path-analysis-with-interaction-2s-pa-int}}

The 2S-PA method, as proposed by Lai and Hsiao in 2021, introduces a refined approach to addressing measurement error within the context of multiple congeneric items by incorporating reliability adjustment. It is similar to RAPI but uses factor scores as single indicators to latent predictors. A key advancement of the 2S-PA approach is its capacity to assign observation-specific estimated reliability, thereby extending its applicability to ordered categorical items and accommodating distributions that deviate from normality (Lai \& Hsiao, 2021; Lai et al., 2023). Besides, conventional SEM models typically estimate measurement and structural models simultaneously, which necessitates a considerable sample size to achieve satisfactory convergence rates (Kline, 2016; Kyriazos, 2018). To address this potential issue, the 2S-PA separates the step of specifying the measurement model from estimating the structural model, therefore alleviating computational burden and improving stability of parameter estimation.

At the first stage of 2SPA, researchers calculate factor scores (\(\hat{F}\)) using first-order indicators for each participant \(j\) for \(j = 1, 2, ..., n\). Next, parallel to RAPI, the factor scores of latent predictors are multiplied to construct a PI for the interaction term \(\xi_{x_{j}}\xi_{m_{j}}\):

\begin{align}
    \begin{bmatrix}
        \widehat{F}_{x_{ij}} \\ 
        \widehat{F}_{m_{ij}} \\
        \widehat{F}_{xm_{ij}}
    \end{bmatrix} = 
    \begin{bmatrix}
        \tau_{x_{ij}} \\
        \tau_{m_{ij}} \\ 
        \tau_{xm_{ij}} 
    \end{bmatrix} + 
    \begin{bmatrix}
        \lambda_{x_{ij}} & 0 & 0 \\
        0 & \lambda_{m_{ij}} & 0 \\ 
        0 & 0 & \lambda_{xm_{ij}} 
    \end{bmatrix} 
    \begin{bmatrix}
        \xi_{x_{j}} \\
        \xi_{m_{j}} \\
        \xi_{x_{j}}\xi_{m_{j}}
    \end{bmatrix} +
    \begin{bmatrix}
        \delta_{x_{ij}} \\
        \delta_{m_{ij}} \\ 
        \delta_{xm_{ij}}
    \end{bmatrix},
\end{align}
wherein the factor scores \(\hat{F}_{x_{j}}\), \(\hat{F}_{m_{j}}\) and the PI \(\hat{F}_{xm_{ij}}\) are single indicators of the respective latent variables. The intercepts, factor loadings, and error variances are all model parameters to be freely estimated.

Although there are multiple ways of calculating factor scores (e.g., regression factor scores, expected-a-posterior factor scores; Devlieger et al., 2016; Estabrook \& Neale, 2013), we used Bartlett factor scores because they align with the strengths of the 2S-PA framework (Bartlett, 1937). Bartlett scores produce unbiased estimates of true factor scores, enhancing the accuracy of our representation of the latent constructs and minimizing distortions that could arise from measurement errors (Hershberger, 2005). As a maximum likelihood-based procedure, it aligns with the estimation methods often used within SEM and with 2S-PA itself. This consistency further strengthens the reliability and interpretability of our analysis. In the case of using Bartlett scores as the factor scores, the Bartlett scores are adjusted to have the same units as latent variables, which means that the factor laodings relating the single indicators to latent variables are constrained to 1 (i.e., \(\lambda_{x_{ij}} = \lambda_{m_{ij}} = \lambda_{xm_{ij}} = 1\)).

Given that the focus of the current study is continuous variable, first-order indicators of \(\xi_{x}\) and \(\xi_{m}\) are continuous variables assumed to be normally distributed, and hence the corresponding error variances are constant for all observations. Accordingly the observation-specific subscript \(j\) from the above equations can be dropped in this study. The error variance constraint for the first-order latent predictors is \(\sigma_{F_{i}}^2\) as \(\sigma_{F_{i}}\) is the estimated standard error of measurement. The error-variance constraint for the interaction term is defined similarly as equation (9). Alternatively speaking, the RAPI method is a special case of 2SPA where the composite scores are used for continuous items (Lai \& Hsiao, 2021).

We argue that the 2S-PA-Int approach is a good alternative to existing methods of estiamting latent interaction effects for its simplicity in model complexity and clarity in model specification. Lai and Hsiao (2021) demonstrated that 2S-PA provides robust and precise estimates with less SE bias, lower Type I error rate, and higher convergence rates in small sample size and low reliability conditions. Hence we expect the 2S-PA-Int method to inherit the advantages and demonstrate desirable performance in latent interaction estimation.


\end{document}
